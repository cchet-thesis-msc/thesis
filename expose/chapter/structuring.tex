\section{Structuring}
\label{sec:thesis-structuring}
This sections deals with the structuring and the extend of the master thesis.
\begin{numbered}
	\item Abstract \textbf{\textit{(1 page)}}
	\item Introduction \textbf{\textit{(2 pages)}}
	\begin{numbered}
		\item Motivation
		\item Objectives
	\end{numbered}
	\item Infrastructure as Code (IaC) \textbf{\textit{(\textasciitilde 7 pages)}}
	\begin{numbered}
		\item The need for IaC
		\item Core concepts
	\end{numbered}
	\item Containerization with Docker \textbf{\textit{(\textasciitilde 7 pages)}}
	\begin{numbered}
		\item The need for containerization
		\item Core concepts
		\item IaC with Docker Compose
		\item Virtualization vs Containerization
	\end{numbered}
	\item Container as a Service (CaaS) with Kubernetes \textbf{\textit{(\textasciitilde 7 pages)}}
	\begin{numbered}
		\item Need for container orchestration
		\item Core concepts
		\item IaC with kubernetes templates
		\item VM orchestration vs container orchestration
	\end{numbered}
	\item PaaS with Openshift \textbf{\textit{(\textasciitilde 7 pages)}}
	\begin{numbered}
		\item Need for application orchestration
		\item Core concepts
		\item IaC with openshift templates
		\item Development and release lifecycle integration
		\item Secret and configuration management
		\item Application orchestration vs container orchestration
	\end{numbered}
	\item Enterprise Service Bus (ESB) \textbf{\textit{(\textasciitilde 5 pages)}}
	\begin{numbered}
		\item Need for an ESB
		\item Core concepts
		\item Challenges of hosting an ESB 
	\end{numbered}
	\item Prototype ESB in Openshift \textbf{\textit{(\textasciitilde 10 pages)}}
	\begin{numbered}
		\item Source code management
		\item Configuration management
		\item Build and release management
		\item ESB Infrastructure in Openshift
	\end{numbered}	
	\item Conclusion \textbf{\textit{(1 page)}}
	\item Prospects \textbf{\textit{(1 page)}}
\end{numbered}