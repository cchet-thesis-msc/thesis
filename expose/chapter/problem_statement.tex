\section{Problem statement}
\label{sec:statement-of-the-problem}
Large enterprises work with several independent applications and systems which have the need to inter exchange data. The data is represented in several formats and versions, which increases the complexity of the communication between the applications and systems. Other challenges are the configuration management and the maintenance of the integrations of the applications and systems into each other, which affects the whole application life cycle.

The concepts of DevOps, PaaS, IaaS and microservice architecture differ to the classic development life cycle, release management and a server infrastructure where DevOps brings development and IT operations together and the classic server infrastructure will be replaced by Platform as a Service (PaaS) or Infrastructure as a Service (IaaS) or more likely both. 

An Enterprise Service Bus (ESB) is an enterprise pattern which in large enterprises is often used to implement applications which connects other applications and systems together. There are several frameworks out there such as JBoss Fuse\cite{redHatJBossFuse}, which is based on a JBoss application server, which helps to design, implement and run an ESB application. Openshift is a open source cloud system where Gepardec assumes it could be used to realize an ESB application or at least several aspect of it. The idea is to split a monolithic ESB application into microservices and host them in an Openshift cloud, where Openshift provides features for monitoring, deploying, scaling and securing microservices.

The following questions shall be answered by this thesis:
\begin{enumerate}
	\item How can components, internal and external microservices, be integrated in an Openshift cloud with as little configuration as possible ?
	\item How can different versions and stages of microservices be managed in an Openshift cloud ?
	\item How to secure microservices hosted in an Openshift cloud ?
\end{enumerate}
Along with using a cloud environment such as Openshift changes will have to be made in the the application life cycle and interaction between the departments and service providers, because the classic approaches will not apply anymore when microservices are hosted in an cloud environment such as Openshift.

