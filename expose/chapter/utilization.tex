\section{Utilization of the state of the art}
\label{sec:deduction}
It will be analyzed how an ESB application can be designed with a microservice architecture and how it can be integrated in Openshift. If possible an ESB specific aspect or functionality will be implemented with Openshift provided mechanisms. The development and release life cycle should be almost fully integrable in Openshift where a single build server environment will become obsolete, but nevertheless Jenkins is fully integrated in Openshift and can be used to configure an Openshift cluster.

This strong integration of the development and release life cycle in Openshift will require changes to be made in the management of the source code, the build definitions, tests and the application life cycle of an application hosted in an Openshift cloud. The separation of an ESB application into isolated services with their own development and release life cycle will bring the possibility of partial releases, but will also increase the complexity when it comes to the interaction between the services.

While the management of an ESB application, which is separated into isolated services, increases in complexity, the complexity of the build, release and necessary infrastructure  of a single service will become less complex. Instead of spending time with a hard to maintain and hard to evolve infrastructure, more time can be spent for management and organization of the application life cycle which becomes more important and complex, when there are several independent services involved.

Another major aspect is security and how secrets are accessed by developers and applications. Openshift contains a secret management where secrets of different types such as SSH-Authentication, Basic-Authentication or simple String data can be managed. These secrets can be injected as environment variables or files into the service Docker containers to provide the secret for a service process. With Openshift the management of the security can be completely separated from the application development and referenced secrets can be represented by placeholders (e.g. environment variables) which get injected by Openshift during container start. During development time, no developer needs access to any secret anymore.