\chapter{Implementation ESB in Openshift}
\label{cha:esbi}
This chapter will discuss the implemented prototype, which has been designed in Chapter \vref{cha:esboc}. The implemented prototype uses lots of Java Enterprise Frameworks and Java Enterprise-Platform specifications, which are beyond the scope of this thesis, therefore a focus will be set on the implementations of the aspects discussed in Section \vref{sec:esboc-aspects}. Is is assumed, that the reader is familiar with Java Enterprise Development, Java Enterprise Frameworks, Maven and the microservices architecture. The implemented prototype is available on Github \footnote{https://github.com/cchet-thesis-msc/prototype}. The repository contains a \mentionedtext{README.adoc} file, which describes how to setup the prototype on a Windows Host. 
\\ \\
The integration services are implemented as microservices, with their own life cycle, and run as standalone applications in Docker Containers on the Openshift Cluster. The integration services communicate via REST with each other, whereby each service provides a proper managed public API. The code bases of the integration services are managed separately, which completely de-couples the integration services from each other.  
\\ \\
As the prototype illustrates, the ESB is represented by an Openshift Project on an Openshift Cluster, whereby the Openshift Cluster acts as the platform for the hosted integration services, and the Openshift Project represents the ESB application. The integration services will be hosted in an Openshift Project, whereby the Openshift Project provides features as discussed in Section \vref{sec:paas-openshift-project} for managing the life cycle of the hosted integration services. The implemented resources for managing the Openshift Project are discussed in Section \vref{sec:esbi-openshift}.
\\ \\
The following Section \vref{sec:esbi-technolody-fis} will briefly introduce the used technologies and frameworks for implementing the integration services with the microservice architecture.

\section{Microservice Technologies}
\label{sec:esbi-technolody-fis}
The following sections will give a brief introduction about the most important technologies and frameworks, used to implement the integration services as microservices. Each implemented integration service is setup the same way, because the concrete purpose of the service does not matter, when the microservices have to be integrated into a distributed service network. All of the following technologies and frameworks provide all necessary API and implementations for implementing a microservice, which is hosted on a Openshift Cluster.

\subsection{JBoss Fuse Integration Services 2.0}
\label{sec:esbi-technology-fis}
JBoss Fuse Integration Services 2.0 is a set of tooling for developing integration services running on a Openshift Cluster. It provides Openshift integrations for different frameworks such as Spring Boot, Karaf or Camel. The services are started via an Java-Agent such as Prometheus or Jolokia, which are used to monitor the service during runtime. Additionally, a Maven Plugin is provided, which allows to interact with the Openshift Cluster during a Maven build, whereby the integration service life cycle on an Openshift Cluster can be managed via Maven Goals. JBoss Fuse Integration Services 2.0 allows developers to interact with a Openshift Cluster in a way like developers did before with an application server like Wildfly \cite{Prometheus2018, Jolokia2018}.

\subsection{Wildfly Swarm}
\label{sec:esbi-technology-swarm}
Wildly Swarm is the Java Enterprise answer to Spring Boot, and is a framework, which allows to package an application into an Uber-JAR. An Uber-JAR is a packaged standalone application, which can be started with the command \inlineJava{java -jar}. During the packaging, only those components of an application server are packaged, which are referenced and needed by the application. The application can then be started via \inlineBash{java -jar app.jar}, whereby the application server is bootstrapped programmatically. The Uber-Jar is a repackaged Java Web-Archive, which could be hosted in any application server environment, which provides all of the referenced dependencies, which are added during the repackaging  \cite{WildflySwarm2018}. 
\\ \\
During the implementation of the prototype Wildfly Swarm has been renamed to Thorntail.io\footnote{https://thorntail.io/}, whereby the namespace has been renamed from \mentionedtext{org.wildfly.swarm} to \mentionedtext{io.thorntail} as well.

\subsection{Fabric8}
\label{sec:esbi-technology-f8}
Fabric8 is an integrated development platform for developing applications on Kubernetes. Fabric8 provides the Maven Plugin for the JBoss Fuse Integration Services 2.0, and focuses on building Docker Images, managing Kubernetes or Openshift resources and deploying Java applications on Kubernetes or Openshift Clusters \cite{Fabric82018}.
\\ \\
The following sections will discuss the implementations of the microservice aspects as discussed in Section \vref{sec:esboc-aspects}.

\section{Security}
\label{sec:esbi-security}
The integration services are secured with OAuth, and authenticate their clients via Keycloak. Keycloak is used as the authentication service, and is a very popular open source identity and authentication application. Wildfly Swarm provides an integration into Keycloak via the Keycloak Adapter, which needs to be added as a dependency to the Maven \mentionedtext{pom.xml}, and to be configured what resources to protect.

\subsection{Service}
\label{sec:esbi-security-service}
This section will discuss the implementation of the security in the service implementations. Listing \vref{ls:esboi-security-pom} shows the dependency, which brings in the Keycloak Adapter. The Keycloak Adapter integrates itself into the Java Web-Security mechanisms, and can therefore be configured with Java Web-Security security constraints.

\begin{listing}[h]
	\xmlFile{\sourceDir/maven-keycloak-swarm.xml}
	\caption{Wildfly Swarm Keycloak-Adapter dependency in pom.xml}
	\label{ls:esboi-security-pom}
\end{listing}

Listing \vref{ls:esboi-security-yaml} shows an excerpt of the Wildfly Swarm configuration file \mentionedtext{project-stages.yml}, which configures the security constraints for the REST Endpoints.

\begin{listing}[h]
	\yamlFile{\sourceDir/project-stages-security.yml}
	\caption{Security configuration in project-stages.yml}
	\label{ls:esboi-security-yaml}
\end{listing}

The following two listings are excerpts of the \mentionedtext{deployment.yml} Openshift Template, which is managed in the integration service code base. Listing \vref{ls:esboi-security-oc-deployment-volume-secret} shows the specification of the secret injection into a Docker Volume. The secrets are injected as files, whereby the file name represents the secret key and the file content represents the secret value. Therefore, that the secrets are managed externally, the developers need to provide the secret name for the service deployment configuration. In this case an expression is used, which can be replaced by Maven Properties, whereby the Maven Properties can be provided in the \mentionedtext{pom.xml} or provided/overwritten by Java Options during the Maven Build process.

\begin{listing}[h]
	\yamlFile{\sourceDir/deployment-volume-secret.yml}
	\caption{Configuration of the secret injection in deployment.yml}
	\label{ls:esboi-security-oc-deployment-volume-secret}
\end{listing}

Listing \vref{ls:esboi-security-oc-deployment-volume-mount} show the specification of the mount of the Docker Volume, which provides the secrets. The mount path is also represented by a Maven Property, because this path is also used in the \mentionedtext{project-stages.yml} file, where it points to the integration service configuration source for the productive stage. The secrets consumed by the services are used the same way as non-sensitive configurations, which are discussed in Section \vref{sec:esbi-configuration}.

\begin{listing}[h]
	\yamlFile{\sourceDir/deployment-volume-secret-container.yml}
	\caption{Configuration volume mount in deployment.yml}
	\label{ls:esboi-security-oc-deployment-volume-mount}
\end{listing}

\subsection{Openshift}
\label{sec:esbi-security-openshift}
This section will discuss the Openshift implementation, whereby the implementation is represented by a shell script, which manages the integration service Openshift Secrets. The secrets are managed outside the code bases of the integration services, and are supposed to be maintained by operators.
\\ \\
Listing \vref{ls:esboi-security-oc-secret} shows the Openshift CLI-Commands, which are used to create the Openshift Secrets. The first command creates an Openshift Secret from literal values, which provides the configurations for the sintegration service. The second command creates an Openshift Secret from a file, whereby the filename is the secret key and the file content is the secret value. The secret file is used by the Keycloak Adapter to validate the client OAuth tokens. 
\newpage 

\begin{listing}[h]
	\bashFile{\sourceDir/bash-oc-secret.txt}
	\caption{Openshift CLI command for creating the secret}
	\label{ls:esboi-security-oc-secret}
\end{listing}

This section discussed the implementations, which are necessary to secure integration services hosted on an Openshift Cluster via OAuth with Keycloak. No source code is necessary, only configuration. The following Section \vref{sec:esbi-configuration} will discuss the configuration of the integration services, which can be applied to the security as well, because secrets in Openshift are used in the service implementations the same way as configuration parameters.

\section{Configuration}
\label{sec:esbi-configuration}
The integration services use the MicroProfile Config specification to be configurable for multiple stages by being able to consume configuration parameters from different configuration sources via injection. Developers are bound to the configuration/secret name, keys and their value type. Developers are not bound to the configuration/secret source, which allows to provide configurations/secrets from different sources and for different stages. 

\subsection{Service}
\label{sec:esbi-config-service}
This section will discuss the implementation of the configuration definition and usage. Listing \vref{ls:esboi-config-pom} shows the dependency, which brings in the MicroProfile Config specification, to enable injectable configurations.
 
\begin{listing}[h]
	\xmlFile{\sourceDir/maven-microprofile-config.xml}
	\caption{Wildfly Swarm MicroProfile-Config dependency in pom.xml}
	\label{ls:esboi-config-pom}
\end{listing}

Listing \vref{ls:esboi-config-project-stages-dev} shows the definition of the configuration source in the \mentionedtext{project-stages.yml} for the development stage, whereby the configuration parameter values are provided hard coded.
\newpage

\begin{listing}[h]
	\bashFile{\sourceDir/project-stages-micro-config-dev.yml}
	\caption{Hard coded configuration for development stage}
	\label{ls:esboi-config-project-stages-dev}
\end{listing}

Listing \vref{ls:esboi-config-project-stages-prod} shows the definition of the configuration source in the \mentionedtext{project-stages.yml} for the production stage, whereby the configuration parameter values are loaded via a directory. The directory location is represented by an Maven Property, because its used in multiple configuration files, as already discussed in Section \vref{sec:esbi-security-service}. The MicroProfile Config specification will load files in this directory by using the filename as the key and the file content as the value.

\begin{listing}[h]
	\yamlFile{\sourceDir/project-stages-micro-config-prod.yml}
	\caption{External configuration for production stage}
	\label{ls:esboi-config-project-stages-prod}
\end{listing}

Listing \vref{ls:esboi-config-inject} shows the injection of the Keycloak Secrets into a CDI Bean, whereby the source of the configuration is unknown. The injected configuration properties are retrieved from a Openshift Secret, but are used in the source code the same way as configurations.

\begin{listing}[h]
	\javaFile{\sourceDir/java-config-inject.java}
	\caption{Injection of Keycloak configuration parameters}
	\label{ls:esboi-config-inject}
\end{listing}

\subsection{Openshift}
\label{sec:esbi-config-openshift}
The Openshift implementation has already been covered by Section \vref{sec:esbi-security-openshift}, because all of the configurations are managed as Openshift Secrets, because they contain sensitive data. 

\section{Tracing}
\label{sec:esbi-tracing}
The integration services use the MicroProfile OpenTracing specification to provide tracing data to a central tracing service. Jaeger\footnote{https://www.jaegertracing.io/} is used as the tracing service, which collects all tracing data and provides a GUI for analyzing the collected traces. 

\subsection{Service}
\label{sec:esbi-tracing-service}
This section will discuss the implementation of the service tracing. Listing \vref{ls:esboi-tracing-pom} shows the dependency, which brings in the MicroProfile OpenTracing specification to enable tracing. 

\begin{listing}[h]
	\xmlFile{\sourceDir/maven-microprofile-opentracing.xml}
	\caption{Wildfly Swarm MicroProfile-OpenTracing dependency in pom.xml}
	\label{ls:esboi-tracing-pom}
\end{listing}

Listing \vref{ls:esboi-tracing-project-stages} shows the configuration  in the \mentionedtext{project-stages.yml} for the integration into the Jaeger tracing service, whereby the configuration parameters are provided by Maven Properties, environment variables and literals. The configuration properties are created as System Properties by Wildfly Swarm, whereby  expressions like \mentionedtext{\$\{env.JAEGER\_PORT\}} are resolved during startup.

\begin{listing}[h]
	\yamlFile{\sourceDir/project-stages-opentracing.yml}
	\caption{Configuration for integration into Jaeger in project-stages.yml}
	\label{ls:esboi-tracing-project-stages}
\end{listing}

Listing \vref{ls:esboi-tracing-java} shows a class which is annotated with \inlineJava{@Traced} on class level, which enables tracing for all methods within this class. The annotation \inlineJava{@Traced} enables an interceptor, which implements the tracing logic. 
\\ \\
A trace is a set of so called spans, whereby a span represents one call in a call chain and contains meta-data of the call such as call duration. The interceptor creates a new span for each call and appends the created span to an existing parent span, or the created span is the parent span. 

\begin{listing}[h]
	\javaFile{\sourceDir/java-tracing.java}
	\caption{Enable tracing for a class}
	\label{ls:esboi-tracing-java}
\end{listing}

\subsection{Openshift}
\label{sec:esbi-tracing-openshift}
The communication between the integration services and tracing service is done via UDP protocol, and therefore Openshift does need any special configuration. Openshift does not interfere with outgoing connections, only incoming.

\section{Logging}
\label{sec:esbi-logging}
The integration services provide logging to a central log aggregation service. Graylog\footnote{https://www.graylog.org/} is used as the log aggregation service, which collects all logging data and provides a GUI for analyzing the aggregated logs.

\subsection{Service}
\label{sec:esbi-logging-service}
This section will discuss the implementation of the service logging. Listing \vref{ls:esboi-logging-pom} shows the dependencies, which bring in the logging implementations. SLF4J\footnote{https://www.slf4j.org/} has been chosen as the logging facade, whereby an integration into Wildfly Swarm used JBoss Logging is provided by SLF4J.

\begin{listing}[h]
	\xmlFile{\sourceDir/maven-swarm-logging.xml}
	\caption{Wildfly Swarm logging dependencies in pom.xml}
	\label{ls:esboi-logging-pom}
\end{listing}

The following listings are part of the \mentionedtext{project-stages.yml} configuration file and configure logging for different stages. Listing \vref{ls:esboi-logging-format-project-stages} shows the configuration of the logging format, which uses Mapped Diagnostic-Context (MDC) parameters to mark a log entry with the transaction id. The configured formatter is used for all stages, because it has been defined globally.

\begin{listing}[h]
	\yamlFile{\sourceDir/project-stages-logging-format.yml}
	\caption{Logging format configuration in project-stages.yml}
	\label{ls:esboi-logging-format-project-stages}
\end{listing}

Listing \vref{ls:esboi-logging-dev-project-stages} shows the logging configuration for the development stage.

\begin{listing}[h]
	\yamlFile{\sourceDir/project-stages-logging-dev.yml}
	\caption{Logging configuration for development stage in project-stages.yml}
	\label{ls:esboi-logging-dev-project-stages}
\end{listing}

Listing \vref{ls:esboi-logging-prod-project-stages} shows the configuration in the \mentionedtext{project-stages.yml} for the logging of the production stage, where the service is contributing its logs to a central log aggregation service. A Syslog Logging-Handler is configured, which sends the logs to the log aggregation service via the Syslog\footnote{https://tools.ietf.org/html/rfc5424} protocol. The configuration where to send the logs, is provided via System Properties, which are created during startup out of environment variables, which are set with values of an Openshift Secret.
\\ \\
Listing \vref{ls:esboi-logging-java-transaction-id} shows the implementation of the interface \mentionedtext{ContainerRequestFilter}, provided by the JAX-RS specification, which is used to capture the trace transaction id on a REST Endpoint. The implementation is depending on the Uber MicroProfile-OpenTracing implementation, because the specification itself does not provide an accessor for the transaction id yet. The captured transaction is set into MDC, where the formatter defined in Listing \vref{ls:esboi-logging-format-project-stages} references the set MDC parameter.
\newpage

\begin{listing}[h]
	\yamlFile{\sourceDir/project-stages-logging-prod.yml}
	\caption{Configuration of the logging for production stage}
	\label{ls:esboi-logging-prod-project-stages}
\end{listing}

\begin{listing}[h]
	\javaFile{\sourceDir/java-logging-tracing-id.java}
	\caption{Capture of tracing id on REST Endpoint}
	\label{ls:esboi-logging-java-transaction-id}
\end{listing} 

Listing \vref{ls:esboi-logging-java-producer} shows the CDI Producer method, which provides the logger instances for injection points. The logger is produced for the Dependent Scope, which means that the life cycle of the logger is managed by the object, which gets the logger injected.
\newpage

\begin{listing}[h]
	\javaFile{\sourceDir/java-logging-producer.java}
	\caption{CDI Producer for dependent scoped logger instances}
	\label{ls:esboi-logging-java-producer}
\end{listing} 

Listing \vref{ls:esboi-logging-java} shows a class using an injected logger to log a info message. As this examples illustrates, the user code has no knowledge about a log aggregation back-end or about a transaction id.

\begin{listing}[h]
	\javaFile{\sourceDir/java-logging.java}
	\caption{Logger usage}
	\label{ls:esboi-logging-java}
\end{listing} 

\subsection{Openshift}
\label{sec:esbi-logging-openshift}
The integration services send their logs to a central log aggregation service via the UDP protocol, and therefore there are no special settings for Openshift necessary. Nevertheless, logs send to the console are collected by Openshift and can be analyzed in the Openshift Web-Console.

\section{Fault Tolerance}
\label{sec:esbi-fault}
The integration services use the MicroProfile FaultTolerance specification to define fault tolerance behavior on methods. Hystrix\footnote{https://github.com/Netflix/Hystrix/wiki/How-it-Works} is a popular framework for failure handling in applications, was the inspiration for the MicroProfile FaultTolerance specification, and is used as the back-end for the Wildfly Swarm provided dependency.

\subsection{Service}
\label{sec:esbi-fault-service}
The integration services use the MicroProfile Fault-Tolerance specification to define the service fault tolerance behavior, which defines the service resilience. Listing \vref{ls:esboi-fault-tolerance-pom} shows the dependency, which brings in the fault tolerance implementations.

\begin{listing}[h]
	\xmlFile{\sourceDir/maven-microprofile-fault-tolerance.xml}
	\caption{Wildfly Swarm MicroProfile-FaultTolerance dependency in pom.xml}
	\label{ls:esboi-fault-tolerance-pom}
\end{listing}

Listing \vref{ls:esboi-fault-tolerance-java} shows the CDI Producer method for producing the Keycloak token, which defines fault behavior for this method. The special use of the method as a CDI Producer method does not affect the fault tolerance logic. Each time when the producer method is called a token request is send to Keycloak, to retrieve an access token. The invocation is retried 5 times with a 100 millisecond delay, and each invocation is timed out after 5 seconds.

\begin{listing}[h]
	\javaFile{\sourceDir/java-fault-tolerance.java}
	\caption{Fault tolerance definition on CDI Producer method}
	\label{ls:esboi-fault-tolerance-java}
\end{listing} 

\subsection{Openshift}
\label{sec:esbi-fault-openshift}
The fault tolerance behavior as discussed in Section \vref{sec:esbi-fault-service} only affects the service itself and not Openshift. But, Openshift provides a kind of fault tolerance, for instance by restarting crashed Pods.
\newpage 

\section{REST API-Management}
\label{sec:esbi-api}
The integration services use Swagger\footnote{https://swagger.io/} to provide documentation for their REST API. Swagger provides an intermediate format, which can be used by tooling for testing and client generation. The REST API represents the public view of the integration service, which is implemented in a way, so that it is de-coupled from the Service Logic, and supports several ways to perform API migrations.
\\ \\
The following sections will discuss the implementation of the REST API-Management on the service side and the REST API usage on the client side. Both use Swagger, whereby the service provides Swagger Documentation and the client uses the Swagger Documentation to generate REST Clients.

\subsection{Service}
\label{sec:esbi-api-service}
This section will discuss the management and implementation of the service REST API. Listing \vref{ls:esboi-api-service-pom} shows the integration service dependencies, which bring in the 
\begin{itemize}
	\item the Java BeanValidation implementation, 
	\item the JAX-RS Server implementation,
	\item the JAX-RS Server Java-BeanValidation integration,
	\item and the Swagger implementations. 
\end{itemize}

\begin{listing}[h]
	\xmlFile{\sourceDir/maven-swagger-service.xml}
	\caption{Wildfly Swarm JAX-RS/Swagger dependencies in pom.xml}
	\label{ls:esboi-api-service-pom}
\end{listing}

Listing \vref{ls:esboi-api-swagger-conf} shows the \mentionedtext{swarm.swagger.conf} configuration file, which configures Swagger for the documented integration service. During startup, Swagger will scan the configured packages for interfaces and classes, which provide documentation in form of Swagger Annotations. The scanned documentations are written to a file names \mentionedtext{swagger.json}, which contains the Swagger Documentation of the integration service REST API.    

\begin{listing}[h]
	\yamlFile{\sourceDir/swarm.swagger.conf}
	\caption{Swagger configuration in swarm.swagger.conf}
	\label{ls:esboi-api-swagger-conf}
\end{listing}

Listing \vref{ls:esboi-api-swagger-java} shows an interface, which specifies an REST Endpoint via JAX-RS Annotations, and provides documentation via Swagger Annotations. Additionally, Java BeanValidation-Annotations are used to define constraints for the input arguments of the REST Operations, so that validation is applied on all incoming requests. The JAX-RS, Java BeanValidation and Swagger Annotations are scanned and applied to the generated Swagger Documentation.

\begin{listing}[h]
	\javaFile{\sourceDir/java-swagger.java}
	\caption{JAX-RS interface with Swagger Annotations}
	\label{ls:esboi-api-swagger-java}
\end{listing}

\subsection{Client}
\label{sec:esbi-api-client}
This section will discuss the implementation of the client, which uses the \mentionedtext{swagger.json} file for generating a REST Client for the given swagger Documentation. The following listings will show the configuration of the Maven Plugins, which are used generate a REST Client out of a Swagger Documentation during a Maven Build.
\\ \\
Listing \vref{ls:esboi-api-client-add-sources-pom} shows the configuration of the Maven Helper-Plugin in the client \mentionedtext{pom.xml} , which is used to add the generated REST Client sources for the compilation. The source directory points to the directory, where the generated REST Client sources are located.
\newpage 

\begin{listing}[h]
	\xmlFile{\sourceDir/maven-swagger-client-add-sources.xml}
	\caption{Maven Helper-Plugin configuration in pom.xml}
	\label{ls:esboi-api-client-add-sources-pom}
\end{listing}

Listing \vref{ls:esboi-api-client-clean-pom} shows the configuration of the Maven Clean-Plugin in the client \mentionedtext{pom.xml} , which is used to clean the unwanted generated sources and resources. The Swagger Maven-Plugin generates a standalone REST Client, which cannot be turned off, but only the plain generated models and interfaces are wanted, without any back-end. 

\begin{listing}[h]
	\xmlFile{\sourceDir/maven-swagger-client-clean.xml}
	\caption{Maven Clean-Plugin configuration in pom.xml}
	\label{ls:esboi-api-client-clean-pom}
\end{listing}

Listing \vref{ls:esboi-api-client-swagger-plugin-pom} shows the Swagger Maven-Plugin configuration in the client \mentionedtext{pom.xml}, which is used to generate the REST Client during Maven Build. Custom Swagger Code-Generator templates are used, due to the fact that there is no Swagger Code-Generator, which only generates plain JAX-RS interfaces. 
\newpage

\begin{listing}[h]
	\xmlFile{\sourceDir/maven-swagger-client-swagger-plugin.xml}
	\caption{Swagger Maven-Plugin configuration in pom.xml}
	\label{ls:esboi-api-client-swagger-plugin-pom}
\end{listing}

Listing \vref{ls:esboi-api-client-api-java} shows the generated JAX-RS interface, which was generated out of the the Swagger Documentation, which was provided by the generated \mentionedtext{swagger.json} file. The generated JAX-RS interface is very similar to the original JAX-RS interface, but could contain differences, for instance, when a REST Operation parameter of type string was documented as value type number.

\begin{listing}[h]
	\javaFile{\sourceDir/java-swagger-client.java}
	\caption{Maven Clean-Plugin configuration in pom.xml}
	\label{ls:esboi-api-client-api-java}
\end{listing}

Listing \vref{ls:esboi-api-client-api-builder-java} shows how to build an REST-Client for the generated JAX-RS interfaces, whereby developers work with the generated API, and the logic for handling the request and response is provided by RESTEasy\footnote{https://resteasy.github.io/}. Changes made to the REST API or its Swagger Documentation will cause compile errors, therefore the usage of the REST Client is type safe. The MicroProfile OpenTracing specification provides a JAX-RS Client-Filter, which integrates the REST Client request into the configured tracing back-end. This integration ensures that calls to another integration service, made via an REST Client, are part of an existing transaction or are the start or a new transaction.

\begin{listing}[h]
	\javaFile{\sourceDir/java-swagger-client-builder.java}
	\caption{Example of building a type safe REST Client}
	\label{ls:esboi-api-client-api-builder-java}
\end{listing}

In the prototype, the built REST Clients are injectable into CDI Beans, whereby the Rest Clients are managed by a custom proxy to apply proper fault tolerance behavior to the REST Client method calls.

\subsection{Openshift}
\label{sec:esbi-api-openshift}
The REST API-Management and migration does not affect Openshift, because the services REST API is either accessible only within the Openshift Project network, or is exposed via a single Openshift Route. For an exposed service, additional Openshift Routes could be created, which for instance redirect calls made to an REST API-Version to another REST API-Version. 
\newpage

\section{Openshift Project}
\label{sec:esbi-openshift}
This section will discuss the implementation of the Openshift Project, which represents the ESB. The implementations are represented by scripts, configurations and secrets, which ensure that the Openshift Project is properly setup and provides all resources consumed by the services, such as Openshift Volumes, Openshift ConfigMaps and Openshift Secrets. The Openshift resources, which are consumed by the hosted integration services, are managed by one script per integration service. 
\\ \\
The scripts, configurations and secrets are managed by operators, which ensure that the Openshift Projects are properly setup and provide all Openshift resources for the hosted integration services for a specific stage, the Openshift Project represents.

\subsection{Scripts}
\label{sec:esbi-openshift-secrets}
This section will discuss the implemented scripts for managing the Openshift resources consumed by the services. Listing \vref{ls:esboi-openshift-oc-service} shows an excerpt of an implemented script, which manages Openshift Secrets created from files for a service. The Openshift Secrets could also have been created from Openshift Templates, whereby the secrets are either hard-coded in the Openshift Templates or are provided via Openshift Template-Parameters.

\begin{listing}[h]
	\bashFile{\sourceDir/bash-oc-service.sh}
	\caption{Shell functions for managing Openshift Secrets via a CLI}
	\label{ls:esboi-openshift-oc-service}
\end{listing}

The scripts are a convenient way for managing Openshift Secrets, and are more flexible then template based Openshift Secret-Management. With Openshift Templates, there would also be the need for scripts, which can create/modify/delete via Openshift Template created resources. The scripts are separated from the actual secrets or configurations, and can therefore be used for all stages, whereby each stage provides their own secrets and configurations. 

\subsection{Templates}
\label{sec:esbi-openshift-config}
There is no need for additional Openshift Templates, when it comes to the integration services, because Fuse Integration Services 2.0 and the Fabric8 Maven-Plugin provide all necessary resources and functionality to manage the integration services via their own code bases. The code bases manage Openshift Templates such as the \mentionedtext{deployment.yml}, which defines the environment for a integration service, by specifying
\begin{itemize}
	\item the injection of configurations and secrets,
	\item the health checks,
	\item the resource limits,
	\item the roll-out behavior,
	\item and the accessible ports.
\end{itemize}

The Fabric8 Maven-Plugin manages only the particular integration service resources, by applying proper labels and accessing the resource by the applied labels.
\\ \\
The following Chapter \vref{cha:esbd} will evaluate and analyze the implemented prototype, if the prototype fulfills the specification of Chapter \vref{cha:esboc} and if the prototype is a valid representation for an ESB in Openshift.


