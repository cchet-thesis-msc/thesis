\chapter{Introduction}
\label{cha:intro}

\section{Motivation}
\label{sec:intro-motivation}
Large enterprises work with several independent applications, where each application covers an aspect of a business of the enterprise. In general, these applications are from different vendors, implemented in different programming languages and with their own life cycle management. To provide a business value to the enterprise, these applications are connected via a network and they contribute to a business workflow. The applications have to interexchange data, which is commonly represented in different data formats and versions. This leads to a highly heterogeneous network of applications, which is very hard to maintain. \\

The major challenge of an IT department is the integration of independent applications into the enterprise application environment. The concept of Enterprise Application Integration (EAI) provides patterns, which help to define a process for the integration of applications into a heterogeneous enterprise application environment. One of these patterns is the Enterprise Service Bus (ESB), which is widely used in the industry \cite{EIP}. \\

Often the term ESB application is used to refer to an ESB, which integrates internal and external hosted applications. But an ESB is a software architectural model, rather than an application. The term could have been established by the usage of middleware such as JBoss Fuse, which provides tooling to integrate applications into an ESB \cite{Fuse2018}. JBoss Fuse is based on the JBoss Enterprise Application Platform (JBoss EAP), where the applications are integrated in a existing runtime environment. \\

With the upcoming of cloud solutions such as Platform as a Service (PaaS) it is now possible to move the platform from a dedicated environment to a cloud environment, where each integration service has its own runtime environment rather than joining an existing runtime environment. The concept of Integration Platform as a Service (IPaaS) relies on top of PaaS and enhances a common PaaS solution with the Integration features needed by EAI \cite{PaaS2015, iPaaSP12015}. \\
   
Thus, enterprises can reduce the effort in implementing and maintaining an ESB, integrating applications into the ESB and reducing the costs of an ESB by using a consumption based pricing model.

\section{Objectives}
\label{sec:intro-objectives}
This thesis aims to implement an ESB on Openshift PaaS \cite{Openshift2018}. Commonly an ESB is implemented with the help of middleware such as JBoss Fuse, which is based on the JBoss EAP. The concepts of PaaS and IPaaS are in general new to the industry, which commonly hosts their integration services in their own data centers, due to the lack of trust for cloud solutions and knowledge about the new approaches such as microservice architecture. \\

A main focus of this thesis is how applications internal and external can be integrated and managed in the PaaS solution Openshift with the ESB pattern. Before implementing an ESB in a PaaS solution such as Openshift, its necessary to understand the new concepts such as Infrastructure as a Service (IaaS), or containerization with Docker, which are covered in the following chapters. The microservice approach and cloud solutions are becoming more important for the software industry. For instance, Red Hat is currently moving its ESB middleware JBoss Fuse to the cloud, where JBoss Fuse will fully rely on Openshift, and the integration services have to be implemented as microservices. This has huge impact on Red Hats customers, who are used to JBoss Fuse on top of JBoss EAP. \\

This thesis was commissioned by the company Gepardec IT Services GmbH, a company that is working in the area of Java Enterprise and cloud development. The migration from a monolithic ESB to a microservice structured ESB, which is hosted in a PaaS environment, is a major concern for them. The migration from a monolithic ESB to a microservice structured ESB will be a major challenge for their customers, because microservice architecture and cloud solutions are mostly new to them. \\

Over the past years, a huge technology dept has been produced by the industry, due to the monolithic architecture and little refactoring work on their applications and hosting infrastructure. It will be hard for them to reduce the produced technology dept, which they will have to, to keep competitive. Gepardec sees a lot of potential for their business and their customers in this new approach of implementing and hosting an ESB.
 



